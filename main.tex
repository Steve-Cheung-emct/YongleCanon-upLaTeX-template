% !TEX program = ptex2pdf
%%		uplatex main
%%		uplatex main
%%		dvipdfmx  main.dvi

\documentclass[永樂大典,整版]{sz}

% 载入自定义设置
\usepackage{stonesettings}

% 載入自定義的字體包
\usepackage{genkofonts}

\prebreakpenalty`︰=10000
\postbreakpenalty`◯=10000

\maintitle{永樂大典}
\subtitle{一萬三千五百六十三}
\author{}
\authorfn{}



%\defineCMYKcolor(0,0.1146,0.2344,0.2471){mypagecolor} % 西厢记古纸色
\defineCMYKcolor(0,0.1709,0.3970,0.2196){mypagecolor} % 稼軒長短句古紙模擬

%%%%%%%%%%%%%%%%%%%%%%%%%%%%%%%%%%%%%%%
% 載入的紙張配置信息。此係必須文件。     %%
\input{pagesizedef.clo}              %%
%%%%%%%%%%%%%%%%%%%%%%%%%%%%%%%%%%%%%%%
%頭注領域の計算%使頭注標識為漢字
\settochuu \kanjichuu %


\def\evyino#1{{\textcolor{konjou}{\fontsize{11}{11}\selectfont\CID{8237}}}\kern-1zw\hbox{\textcolor{yellow}{#1}}}% 陰文 

\def\evyint#1{{\textcolor{konjou}{\fontsize{11}{11}\selectfont\CID{8237}\CID{8237}}}\kern-20pt\hbox{\textcolor{yellow}{#1}}}% 陰文 

\def\yino#1{~\inhibitglue{\textcolor{konjou}{\fontsize{18}{18}\selectfont\CID{8237}}}\kern-1zw\hbox{\textcolor{yellow}{#1}}~\inhibitglue}% 陰文 

\def\yint#1{\inhibitglue{\textcolor{konjou}{\fontsize{18}{18}\selectfont\CID{8237}\CID{8237}}}\kern-34pt\hbox{\textcolor{yellow}{#1}}~\inhibitglue}% 陰文 


\begin{document}

\userelfont\ujlreq
\pagestyle{empty}


%%%%%% 封面 %%%%%%
\pagecolor{konjou}

\maketitle

\cleardoublepage



\pagecolor{gray!5}
%\pagecolor{ikkonzome} % 一斤染
%\pagecolor{sakurairo} % 樱色 少女粉
%\pagecolor{shiou!60}


%%%%%% 正文 %%%%%%
\LARGE 

\pagestyle{plain}

\ugenkomin

% 載入分頁文檔
%\setcounter{page}{-1}
%\input{./CONTEXT/zh1.txt}

% 載入整版文檔
%\setcounter{page}{0}
%\input{./CONTEXT/zh2.txt}

% 載入永樂大典測試文本
\setcounter{page}{1}
\setcounter{ppage}{31}

\ularge
\par{\leavevmode\kern-1.8zw\raise-4mm%
	\hbox{\gtfamily\ebseries{永樂大典}巻之八百三%
	\hspace{3zw}\hbox{\Huge{二支}}}	}

\par{\leavevmode\kern1.5zw\raise-4mm\hbox{{\gtfamily\ebseries{詩}}%
	{\LARGE\gtfamily\hskip0.33zw\raise3mm\hbox{詩話四十五}}}}



\normalsize
\vskip18pt

\par\noindent
\red{千家詩話總龜}~%
句法門\UTF{3000}%
前人文章\gou%
各自一種句法\gou%
如老杜今君起拖春江流\gou%
予亦江邉具小舟\gou%
同心不减骨肉親\gou%
每語見許文章伯\gou%
如此之類\gou%
老杜句法也\gou%
東坡秋水今幾竿之類\gou%
自是東坡句法\gou%
魯直之夏扇日在搖\gou%
行樂亦云聊\gou%
此魯直句法也\gou%
學者若能遍考前作\gou%
自然度越流輩\gou~\dahange{同上}\UTF{3000}%
淵明退之詩\gou%
句法分明卓然異衆\gou%
惟魯直爲能深識之\gou%
學者若能識%
此等語\gou\\%
自然過人\gou%
阮嗣宗詩亦然\gou~\dahange{同前}\UTF{3000}%
徐師川云\gou%
作詩回頭一句\gou%
最爲難道\gou%
如山谷詩所謂\gou%
忽思鍾陵江十里%
之類是也\gou%
他人豈如此\gou%
尤見句法宏壯\gou%
山谷平日詩多用此格\gou%
~\dahange{同前}\UTF{3000}%
徐師川云\gou%
爲詩文\gou%
常患意不屬\gou%
或只得一句語意便盡\gou%
欲足成一章\gou%
又惡其不相稱\gou%
師川云\gou%
但能知意不屬\gou%
則學可進矣\gou%
凡注意作詩文\gou%
或得一兩句而止\gou%
若未有次句\gou%
即不若且休\gou%
養銳以待新意\gou%
若盡力湏要相屬\gou%
譬如力不敵而苦戰\gou%
一敗之後意氣沮矣\gou\UTF{3000}%
王荊公好集句\gou%
嘗於東坡處見古硯\gou%
東坡合荊公集句\gou%
荊公云\gou%
巧匠斲山骨\gou%
只\空行


得一句遂逡巡而去\gou%
山谷嘗有句云\gou%
麒麟臥葬功名谷\gou%
終身不得好對\gou%
~\dahange{同}\\\dahange{上}\UTF{3000}%
莊子文多奇變\gou%
技經肯綮之未嘗\gou%
乃未嘗技經肯綮也\gou%
詩句時有此法\gou%
如昌黎一蛇兩頭見\gou%
未曾拘官計日月\gou%
欲進又不可\gou%
君欲强起時難更\gou\\%
坡云迨兹霜雪未\gou%
兹謀待君必\gou%
聊亦記吾曹\gou%
餘人罕敢用\gou%
~\dahange{黃常明}\UTF{3000}\UTF{3000}%
苦吟門\UTF{3000}山澤之儒多癯\gou%
詩人猶甚\gou%
子美有思君令人瘦\gou%
樂天云形容瘦{\CID{13977}}詩情苦\gou%
豈是人間有相人\gou%
又云\gou%
貌將松共瘦\gou%
心與竹倶空\gou%
李商隱瘦盡東陽姓沈人\gou%
掉頭撚鬚之苦\gou%
豈有張頤豐頬者哉\gou%
沈昭略嘗戯王約以肥而癡\gou\\%
答以瘦而狂\gou%
昭略喜\gou%
曰瘦已勝肥\gou%
狂以勝癡\gou%
~\dahange{黃常明}\UTF{3000}%
賈島詩如鳥從井口出\gou%
人自岳陽來\gou%
貫休此夜一輪滿\gou%
清光何處無\gou%
皆經年方得句\gou%
以見其詞澁思苦\gou%
若非\gou%
好事者誇辭亦繆用其心也\gou%
~\dahange{同上}\UTF{3000}%
後山詩話云\gou%
司空圖善論前人詩\gou%
如謂元白爲力勍氣僝\gou%
乃都會之豪佑郊島\gou%
非附於蹇澁\gou%
無所置才\gou%
皆切中其病\gou%
及自評其作\gou%
乃以南樓山最秀\gou%
北路邑偏清\gou%
爲假令作者復生\gou%
亦當以着題見許\gou%
此殆不可曉\gou%
當局者迷\gou%
故人情之通患\gou%
如樂天所謂\gou%
劚石破山\gou%
先觀鑱迹\gou%
發矢中的\gou%
兼聽弦聲\gou%
使人不見其詩而聞此語\gou\\%
當以爲何如哉\gou%
冷齋夜話云\gou%
賈島詩有影畧句\gou%
韓退之喜之\gou%
其渡桑乾詩曰\gou%
客舍并州三十霜\gou%
歸心日夜憶咸陽\gou%
而今更渡桑乾水\gou%
却望并州是故鄕\gou%

\clearpage

\UTF{3000}

\chapter{千家詩話總龜}

\vspace*{4mm}

\section{千家詩話總龜}

\vspace*{13pt}
\normalsize


\par\noindent
\red{千家詩話總龜}~%
句法門\UTF{3000}%
前人文章\gou%
各自一種句法\gou%
如老杜今君起拖春江流\gou%
予亦江邉具小舟\gou%
同心不减骨肉親\gou%
每語見許文章伯\gou%
如此之類\gou%
老杜句法也\gou%
東坡秋水今幾竿之類\gou%
自是東坡句法\gou%
魯直之夏扇日在搖\gou%
行樂亦云聊\gou%
此魯直句法也\gou%
學者若能遍考前作\gou%
自然度越流輩\gou~\dahange{同上}\UTF{3000}%
淵明退之詩\gou%
句法分明卓然異衆\gou%
惟魯直爲能深識之\gou%
學者若能識%
此等語\gou\\%
自然過人\gou%
阮嗣宗詩亦然\gou~\dahange{同前}\UTF{3000}%
徐師川云\gou%
作詩回頭一句\gou%
最爲難道\gou%
如山谷詩所謂\gou%
忽思鍾陵江十里%
之類是也\gou%
他人豈如此\gou%
尤見句法宏壯\gou%
山谷平日詩多用此格\gou%
~\dahange{同前}\UTF{3000}%
徐師川云\gou%
爲詩文\gou%
常患意不屬\gou%
或只得一句語意便盡\gou%
欲足成一章\gou%
又惡其不相稱\gou%
師川云\gou%
但能知意不屬\gou%
則學可進矣\gou%
凡注意作詩文\gou%
或得一兩句而止\gou%
若未有次句\gou%
即不若且休\gou%
養銳以待

\clearpage
\UTF{3000}
\endinput

\begin{minipage}<y>[htpb]{120mm}
\begin{tikzpicture}
\node [above,]  at%
	 (10,10) {\includegraphics[width=25pt,height=25pt]{1.pdf}};
\end{tikzpicture} %
\end{minipage}



%%%%%% 封底 %%%%%%



\end{document}


